\chapter{Introduction}

Artificial intelligence (AI) techniques have recently started enjoying widespread industry awareness and adoption; the use of AI is increasingly prevalent in all sectors \cite{wirtz2019artificial,bosch2021engineering}. The reasons behind this are manifold \cite{jordan2015machine}, to name a few: recent breakthroughs in deep learning (DL), increased public awareness, an abundance of available data, access to powerful low-cost commodity hardware, education, but most interestingly, the rise of high-level libraries making ready-to-use state-of-the-art (SOTA) models easily available. The latter practically abolishes the barrier of entry for applying AI --- and with that --- can help use cases in various areas. 

However, to achieve robust deployments, the successful integration of AI components into production-ready applications demands strong engineering methods \cite{serban2020adoption}. That is why it is as essential as ever to also focus on the quality and robustness of deployed models and software. For instance, the lack of a proper overview of data transformation steps may lead to suboptimal performance and to introducing unintended biases, which might contribute to the ever-increasing negative externality of misused AI \cite{o2016weapons}.

Concerningly, a peculiar tendency seems to be unfolding: even though industry professionals already have access to numerous frameworks for deploying AI correctly and responsibly, case studies and developer surveys have found that a considerable fraction of deployments does not follow best practices \cite{serban2020adoption,haakman2021ai,amershi2019software,de2019understanding,sculley2015hidden}. Utilising state-of-the-art machine learning (ML) models has become reasonably simple; applying them correctly is as intricate and nuanced as ever. 

This thesis sets out to investigate the reasons behind the apparent asymmetry between industry adoption of accessible AI-libraries and existing reusable solutions for robust AI deployments. It is hypothesised that the primary reason for the underwhelming adoption rate of best practices is the short supply of professionals equally proficient in the domains of both data science and software engineering. Nevertheless, even without their presence, practitioners could rely on frameworks to achieve some level of automation and maturity in their deployment processes. However, the barrier of entry for using such existing libraries is too high, especially when compared with the simplicity of AI-libraries.

Therefore, a software framework --- called \href{https://github.com/schmelczer/great-ai}{\textit{GreatAI}} --- is designed, and its design is presented in this thesis. The principal motivation behind the construction of \textit{GreatAI} is to facilitate the responsible and robust deployment of algorithms and models by designing a more accessible API in an attempt to overcome the practical drawbacks of other similar frameworks. Its name stands for its main aim: to assist easily creating \underline{G}eneral \underline{R}obust \underline{E}nd-to-end \underline{A}utomated, and \underline{T}rustworthy AI deployments.

The utility of \textit{GreatAI} is validated using the principles of design science methodology \cite{wieringa2014design} through iteratively designing its API and implementation in a case study concerning the natural language processing (NLP) pipeline for a commercial product in collaboration with \href{https://scoutinscience.com/}{ScoutinScience B.V.} The goal of the aforementioned software suite is to evaluate technology transfer opportunities in scientific publications. Subsequently, interviews are conducted with practitioners to validate the generalisability of the design.

The choice of case study subject is no coincidence; while working on the ScoutinScience platform for the last two years, my colleagues and I have increasingly noticed the same recurring challenges in deploying and operating AI/ML pipelines. This has motivated me to pursue a general solution. Considering that the company's predominant field is NLP, the case studies, and hence, the prototype of \textit{GreatAI} will also focus primarily on deploying NLP models. Nonetheless, the motivation for creating a general solution for all AI/ML contexts remains and will be taken into account every step of the way.

\section{Research questions}

I hypothesise that facilitating the adoption of AI deployment best practices is viable by finding less complex framework designs that are easier to adopt in order to decrease the negative externality of misused AI. This paper investigates the hypothesis by answering the following research questions. 

\begin{rqlist}
  \item To what extent does the complexity of deploying AI hinder industrial applications?
  \item What API design techniques can be effectively applied in order to decrease the complexity of correctly deploying AI services?
  \item To what extent can \textit{GreatAI} automatically implement AI deployment best practices?
  \item How suitable is the design of \textit{GreatAI} for helping to apply best practices in other contexts?
\end{rqlist}

In this case, complexity refers to the difficulty faced by professionals (Data Scientists and Software Engineers alike) when integrating third-party libraries with their solutions. This could also be described as the barrier of entry or steepness of the learning curve. If the aforementioned hypothesis is correct, the adoption of best practices can be efficiently increased by decreasing this complexity. AI deployment best practices entail the technical steps that ought to be taken to achieve robust, end-to-end, automated, and trustworthy deployments. These are detailed in Section \ref{section:requirements}.

The existence question regarding the problem itself (\textbf{RQ1}) is answered by reviewing the literature of more than 30 published case studies in Chapter \ref{chapter:background}. \textbf{RQ2} and \textbf{RQ3} are closely connected: the design and evaluation phases utilised to answer them follow an iterative process. They are examined in Chapters \ref{chapter:design} and \ref{chapter:case} respectively. The final evaluation step is to ascertain the capability of the framework's design to generalise beyond a single subdomain and problem context. This question, \textbf{RQ4}, is investigated through interviews with industry professionals in Chapter \ref{chapter:interviews}.

\section{Structure}

The rest of the thesis is organised as follows: Chapter \ref{chapter:background} approaches the problem and the state-of-the-art from three perspectives: the recent trends of AI-library API designs, the experiences gained from practical applications, and a comparison of existing deployment options. Next, the methodology utilised for the subsequent chapters is described in Chapter \ref{chapter:methods}. The design cycle is broken into two chapters, Chapter \ref{chapter:design} and \ref{chapter:case}. The former clarifies the scope and describes the design principles, while the latter details the specifics of the practical case study and the framework's interaction with it. The contributions of the novel design and obtained results are shown and further validated by conducting interviews with industry professionals in Chapter \ref{chapter:interviews}. The thesis is concluded in Chapter \ref{chapter:conclusion}.
