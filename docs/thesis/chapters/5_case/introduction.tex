\chapter{The ScoutinScience platform} \label{chapter:case}

The core product of \href{https://scoutinscience.com/}{ScoutinScience B.V.} is its platform. The clients are technology-transfer offices of Dutch and German universities, government organisations (e.g.: Wetsus), and corporates (e.g.: Heraeus Group, Ruma Rubber B.V.) who wish to extend the scope of their R\&D activities. ScoutinScience connects to multiple data sources of academic publications and integrates them into a single database. Each new publication is evaluated with a suite of AI components that ultimately determine its technology transfer potential. Other features are also extracted that help the users get a quick overview of the authors, topics, and contributions of a given piece of research.

Each client organisation gets to see a different filtered view of this database ranked by the predicted probability of technology transfer opportunities being present. The main motivation is to make these business developers' and other professionals work more efficient by showing them which papers have the largest likelihood of being considered interesting by them. 

To achieve this, we have a service-based architecture \cite{kleppmann2017designing} on the backend, apart from the data integration, communication, and business logic, it is made up of services wrapping simpler (phrase-matching, Naïve Bayes) and more sophisticated (conditional random fields, transformer) models. As we will soon see, these can also depend on each other, for instance, based on the predicted scientific domain, a different model can be chosen for scoring certain aspects of papers.

I was among the first engineers on the team which has grown considerably in the past two years. While architecting, designing, and integrating more and better models into our software solution, I experienced the same difficulties as described in Chapter \ref{chapter:background}. The gap between prototypes and production-ready services is larger than it seems. It is also larger than it should be. This motivated me to investigate the state-of-the-art and had found that it is insufficient in many cases. Since the ScoutinScience platform is a quite typical example of applying AI in the industry, it will serve as the real-life case, problem context, and testbed for attempting to design a solution which can advance the state-of-the-art.

In this chapter, the process of designing \textit{GreatAI} is described along with how it fits into real-life use cases. First, a simple experiment is presented which leads to the implementation of a service, subsequently, as the feature-set of the library grows and matures, a more complex software service is developed. Lastly, the final version of the design is presented and qualitatively evaluated to verify how well it satisfies the requirements described in Section \ref{section:requirements}.
